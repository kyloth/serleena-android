%%%%%%%%%%%%%%%%%%%%%%%%%%%%%%%%%%%%%%%%%%%%%%%%%%%%%%%%%%%%%%%%%%%%%%%%%%%%%%%%
%
% Template di sorgente LaTeX per la scrittura di documentazione del gruppo
% Kyloth.
%
%%%%%%%%%%%%%%%%%%%%%%%%%%%%%%%%%%%%%%%%%%%%%%%%%%%%%%%%%%%%%%%%%%%%%%%%%%%%%%%%

% 
% CLASSE DI DOCUMENTO
% ===================
%
% Classe di documento che definisce il layout e il set di macro standard per
% tutti i documenti. Estende la classe "article", quindi tutte le definizioni
% di "article" sono disponibili.
%
\documentclass{kylothdoc}

%
% DEFINIZIONI
% ===========
%
% Nel preambolo del sorgente DEVONO essere definite tutte le seguenti variabili
% a descrizione del documento, pena errore di compilazione.
%

% Titolo del documento.
\doctitle{Norme di progetto}

% Autori del documento. Vi possono essere fino a sette autori, dichiarati con
% i comandi \authorOne, \authorTwo, \authorThree, ..., \authorSeven.
% In ogni caso, è OBBLIGATORIO definire almeno \authorOne.
\authorOne{Un autore}
\authorTwo{Altro autore}

% Il responsabile si dichiara con il comando \manager, e deve SEMPRE essere
% definito.
\manager{Il responsabile}

% Per i verificatori la procedura è simile agli autori: fino a un massimo di
% sette, \verifierOne da definire obbligatoriamente.
\verifierOne{Un verificatore}

% Versione del documento.
\version{1.0.1}

% Per la lista di distribuzione la procedura è simile agli autori e ai
% verificatori: fino a sette elementi di lista, vi deve essere sempre almeno
% l'elemento \distribListOne.
\distribListOne{Kyloth}
\distribListTwo{Prof. Vardanega Tullio}

% Tipologia del documento (uso interno, uso esterno).
\usetype{Interno}

% Breve descrizione del documento.
\summary{Questo documento specifica le norme di progetto adottate da Kyloth.}

% Path dei loghi da usare all'interno del documento. \biglogo identifica il logo
% grande che apparirà in prima pagina, mentre \smalllogo identifica il logo
% presente nell'intestazione di ogni pagina. E' possibile definire opzionalmente
% una scala per l'immagine, che di default è 1.
\biglogo[0.1]{logo.jpg} % file logo.jpg scalato a 10%.
\smalllogo{logo.jpg} % file logo.jpg con dimensioni di default del 100%.

\begin{document}
% Il comando \makedoctitle crea la prima pagina del documento, con tutte le
% informazioni indicate nel preambolo.
\makedoctitle

% La seconda pagina del documento deve contenere il diario delle modifiche.
% Il diario va tenuto in un file XML (ad esempio, history.xml), strutturato
% secondo lo schema XSD. Tale file va poi tradotto in sorgente LaTeX e scritto
% su file durante il processo di build del documento, ad esempio
% history.xml --> history.tex (vedi README in directory xsd/ per guida how-to).
% Il file history.tex va poi incluso nel sorgente principale con \input{history}
% (se il file è pippo.tex, il comando diventa \input{pippo}).
% Il comando \newline prima dell'\input fa sì che il contenuto appaia in una
% nuova pagina (come richiesto dalle norme di progetto).
\newpage
\input{history}

% \tableofcontents stampa l'indice dei contenuti.
\newline\tableofcontents

% In presenza di tabelle o figure (e solo in quel caso), vanno inseriti anche i
% comandi \listoftables e \listoffigures. 
\listoftables
\listoffigures

% Qui il contenuto...

% Comandi per BiBTeX
\bibliography{file-contenente-la-bibliografia}
\bibliographystyle{stile-della-bibliografia}

\end{document}
